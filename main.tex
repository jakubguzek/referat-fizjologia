\documentclass[two column, twoside, a4paper]{article}

\usepackage[utf8]{inputenc}
\usepackage{float}
\usepackage[backend=biber, maxbibnames=3, style=nature, autocite=inline]{biblatex}
\usepackage[polish]{babel}
\usepackage[T1]{fontenc}
\usepackage{fancyhdr}
\usepackage{titlesec}
\usepackage{blindtext}
\usepackage{cuted}
\usepackage{dblfloatfix}
\usepackage{tikz}
\usepackage[most]{tcolorbox}
\usepackage[columnsep = 1cm,
	        lmargin = 0.6in,
	        rmargin = 0.4in,
	        tmargin = 0.5in,
	        bmargin = 0.65in,
	        headsep = \baselineskip]{geometry}

\addbibresource{$BIB}

% Custom commands
\newcommand*\circled[1]{\tikz[baseline=(char.base)]{
            \node[shape=circle,draw,inner sep=2pt, color = orange!90!white] (char) {#1};}}

% Section Formatting
\titleformat{\section}
{\sc \bfseries \Large}
{}
{0em}
{}[\titlerule]

\titleformat{\subsection}
{\bfseries \large}
{}
{0em}
{}

\titleformat{\subsubsection}
{\bfseries}
{}
{0em}
{}

% Box formatting
\tcbset{enhanced, colback=orange!15!white, boxrule = 1pt, coltitle = orange!90!white, colbacktitle = orange!15!white, colframe= orange!50!white}

\pagestyle{fancy}
\fancyhf{}
\fancyhead[RE, LO]{Szkoła Główna Gospodarstwa Wiejskiego}
\fancyhead[LE, RO]{Biotechnologia}
\fancyfoot[RE, LO]{Jakub J. Guzek}
\fancyfoot[LE, RO]{\thepage}
\fancyfoot[CE,CO]{Zmiany w organizmie matki w czasie ciąży i Hormonalna regulacja porodu}
\renewcommand{\footrulewidth}{0.05pt}

\begin{document}

\begin{strip}
{\sc \bfseries \huge \fontfamily{phv}\selectfont Zmiany w organizmie matki w czasie ciąży i

\vspace{2pt} hormonalna regulacja porodu} \vspace{\baselineskip}

{\bfseries \Large Jakub J. Guzek}

{Szkoła Główna Gospodarstwa Wiejskiego, Biotechnologia, Nr. albumu: 195528}\vspace{\baselineskip}

\hrule
\end{strip}

\section{Wstęp}

Ciąża, poród, a także inne procesy związane z rozrodem u ssaków są kontrolowane przez skomplikowany system, w którym ściśle współpracuje ze sobą wiele hormonów i substancji regulujących, przy jednoczesnej współpracy układu nerwowego. Taka regulacja neurohormonalna działająca na wielu poziomach, umożliwia dokładną kontrolę skomplikowanych procesów rozwojowych zachodzących w rozwijającym się zarodku oraz interakcję między procesami zachodzącymi w organizmie matki i dziecka. Wiele z substancji biorących udział w tej regulacji wykazuje działanie wielotorowe na wiele różnych tkanek zarówno u matki, jak i u dziecka i dodatkowo obecne są znaczne różnice w ich działaniu między gatunkami.
Z tego powodu omówienie działania poszczególnych hormonów oddzielnie, zaciera istotę interakcji jakie zachodzą w tych procesach na wielu poziomach -- znacznie korzystniejsze jest omówienie działania tych substancji w ujęciu systemicznym, jako dynamicznego biologicznego systemu, którego struktura ulega znacznym zmianom w czasie. Omówienie jednak tego zagadnienia w ujęciu biologii systemów jest skomplikowane -- należy bowiem pamiętać, że sieć interakcji między hormonami i ich receptorami w czasie ciąży i porodu nie jest odizolowana od innych tego rodzaju sieci w reszcie organizmu oraz że wiele interakcji w takim układzie ma charakter zmienny nie tyko w zależności od czasu ale także od czynników środowiskowych.

\begin{tcolorbox}[title=\hspace{-1.5em} \circled{\textbf{Box 1}} \textbf{\large{Wyjaśnienie skrótów}}]
	\begin{description}
		\item[GnRH] -- gonadoliberyna (\textit{ang. gonadotropin-releasing hormone})
		\item
	\item
	\end{description}
\end{tcolorbox}

\section{Przed ciążą}

Oogeneza rozpoczyna się u samic wielu gatunków ssaków już na etapie życia płodowego. Pierwotne komórki płciowe są po migracji do zawiązków gonad przekształcane w oogonia. Oogonia ulegają w dalszej kolejności wielu podziałom mitotycznym po których rozpoczyna się podział mejotyczny, który ulega zatrzymaniu na etapie profazy mejozy I -- komórka taka nazywa się oocytem I rzędu i może pozostawać w takim stadium przez długi czas aż samica osiągnie dojrzałość płciową. \autocite{Krzymowski2005}

Po osiągnięciu przez samicę dojrzałości płciowej oocyty rosnących pęcherzyków jajnikowych przechodzą dalsze etapy podziału mejotycznego aż do ponownego zatrzymania procesu na etapie metafazy mejozy II \autocite{Sawicki2017}. Dojrzewanie pęcherzyków jajnikowych zachodzi cyklicznie od kiedy samica osiągnie dojrzałość płciową.

\subsection{Regulacja czynności jajnika i dojrzewania pęcherzyków jajnikowych}


\subsection{Owulacja}

\blindtext
\subsection{Regulacja powstawania, zanikania i utrzymywania ciałka żółtego}

\blindtext
\subsection{Zapłodnienie}

\blindtext
\section{Ciąża}

\blindtext
\subsection{Wczesna ciąża}

\blindtext
\subsection{Późna ciąża}

\blindtext
\section{Poród}

\blindtext
\section{Laktacja}

\blindtext
\section{Podsumowanie}

\blindtext

\printbibliography

\end{document}
